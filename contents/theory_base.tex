\chapter[\hspace{0pt}理论基础]{{\heiti\zihao{3}\hspace{0pt}理论基础}}\label{chapter3: 理论基础}

\removelofgap
\removelotgap

为了便于对后续视频源BT.2020色域与普通显示屏RGB色域之间映射关系的分析,我们首先引入标准色度系统的数学模型,对常见色彩空间进行建模表示。这些空间构成了本问题中色彩转换和损失评估的基础框架。

\section[\hspace{-2pt}颜色空间理论]{{\heiti\zihao{4} \hspace{-8pt}颜色空间理论}}\label{section2: 颜色空间理论}

人类对颜色的感知是一个复杂的生理和心理过程。为了量化和描述颜色,引入了颜色空间的概念。本节介绍本文涉及的主要颜色空间及其相互转换关系。

\subsection[\hspace{-2pt}CIE1931标准色度观察者与XYZ颜色空间]{{\heiti\zihao{-4} \hspace{-8pt}CIE1931标准色度观察者与XYZ颜色空间}}\label{subsection2: CIE1931标准色度观察者与XYZ颜色空间}

CIE 1931是由国际照明委员会(CIE)于1931年定义的色彩模型,其核心在于基于实验测量建立的"标准色度观察者"响应曲线。这一模型通过三条匹配函数 $\overline{x}(\lambda),\overline{y}(\lambda),\overline{z}(\lambda)$ 将任意波长下的光谱功率分布(SPD)映射为三刺激值(Tristimulus Values):
\begin{equation}
\begin{aligned}
  &X = \int_{\lambda}S(\lambda)\overline{x}(\lambda)d\lambda,\ \ \ Y = \int_{\lambda}S(\lambda)\overline{y}(\lambda)d\lambda,\ \ \ Z = \int_{\lambda}S(\lambda)\overline{z}(\lambda)d\lambda
\end{aligned}
\end{equation}

CIEXYZ 是一个以三刺激值为基础的线性色彩空间,被视为"设备无关"的色彩表示方式。其三个分量 $(X,Y,Z)$ 分别对应红、绿、蓝三种感知通道。Y 分量也通常用作\textbf{亮度(Luminance)}的代表。该空间是许多其他色彩空间(如 Lab、sRGB、BT.2020)的中间标准基础。通常不同色域之间的转换以此为中介。

\subsection[\hspace{-2pt}CIELab颜色空间]{{\heiti\zihao{-4} \hspace{-8pt}CIELab颜色空间}}\label{subsection2: CIELab颜色空间}

CIELab 空间是基于 CIEXYZ 空间定义的感知均匀色彩空间,能够更好地符合人眼对颜色差异的敏感性。其由以下三个分量构成:
\begin{equation}
\begin{aligned}
  &L^{*}\ ,\ \ a^{*}\ ,\ \ b^{*}\ 
\end{aligned}
\end{equation}
其中,$L^{*}$代表明度,$a^{*}$代表红绿轴,$b^{*}$代表黄蓝轴。具体变换公式如下(以D65白点为例):
\begin{equation}
\begin{aligned}
  &f(t) = 
\begin{cases}
  t^{\frac{1}{3}}\ \ \ \ t>\delta^{3}\\
  \frac{t}{3\delta^{2}}+\frac{4}{29}\ \ \ \ t\leq \delta^{3}\\
\end{cases}
,\ \ \delta=\frac{6}{29}
\end{aligned}
\end{equation}

\begin{equation}
\begin{aligned}
  &L^{*} = 116f(\frac{Y}{Y_{n}})-16,\ \ \ a^{*}=500[f(\frac{X}{X_{n}})-f(\frac{Y}{Y_{n}})],\ \ \ b^{*} = 200[f(\frac{Y}{Y_{n}})-f(\frac{Z}{Z_{n}})]
\end{aligned}
\end{equation}

其中$(X_{n},Y_{n},Z_{n})$ 为参考白点(如D65)的三刺激值。在 Lab 空间中,两点之间的欧氏距离与人眼感知的颜色差异近似成正比,这使其成为颜色差异评估的理想空间。

\subsection[\hspace{-2pt}色度图与色域表示]{{\heiti\zihao{-4} \hspace{-8pt}色度图与色域表示}}\label{subsection2: 色度图与色域表示}

CIEXYZ 空间中颜色可以通过如下变换得到色度图中的坐标:
\begin{equation}
\begin{aligned}
  &x=\frac{X}{X+Y+Z},\ \ y=\frac{Y}{X+Y+Z}
\end{aligned}
\end{equation}

该色度图表示了所有可见光的二维投影范围,设备的色域可以通过其三基色的 $(x,y)$ 点连线形成三角形表示。色域越大,所能表示的颜色越丰富。该图是色彩匹配与色彩损失分析的重要工具。

色域 (Gamut) 是指一个颜色系统或设备能够显示或捕捉的颜色范围。在 CIE 1931 色度图上,色域通常由其基色(原色)的色度坐标连接形成的多边形表示。色域映射 (Gamut Mapping) 是指将一个色域的颜色转换到另一个色域的过程,旨在最小化颜色失真,尤其是在目标色域小于源色域时。

传统的显示系统通常采用三基色 (RGB) 来显示颜色。然而,为了更广阔的色域和更丰富的色彩表现,多基色显示技术(如本问题中的五通道 LED 显示屏)正在兴起。这些系统通过增加额外的基色来扩展其可显示的颜色范围。

\subsection[\hspace{-2pt}常见颜色空间对比]{{\heiti\zihao{-4} \hspace{-8pt}常见颜色空间对比}}\label{subsection2: 常见颜色空间对比}

常见的颜色空间包括:

\begin{itemize}
    \item \textbf{RGB (Red, Green, Blue)}:基于三原色加法混色的颜色模型,常用于显示设备和图像输入设备。然而,RGB 并非感知均匀,即欧氏距离不直接对应人眼感知的颜色差异。
    \item \textbf{XYZ (CIE 1931 XYZ)}:国际照明委员会 (CIE) 定义的一种基于人眼视觉生理特性的颜色空间。它涵盖了人眼可见的所有颜色,且与设备无关。其分量 $X, Y, Z$ 分别对应于光谱在人眼视锥细胞响应曲线下的积分。$Y$ 分量通常表示亮度信息。
    \item \textbf{Lab (CIE L*a*b*)}:一种感知均匀的颜色空间,从 XYZ 空间推导而来。$L^*$ 表示亮度,从黑到白;$a^*$ 表示从绿到红的颜色信息;$b^*$ 表示从蓝到黄的颜色信息。
\end{itemize}

\section[\hspace{-2pt}颜色差异度量理论]{{\heiti\zihao{4} \hspace{-8pt}颜色差异度量理论}}\label{section2: 颜色差异度量理论}

为了量化两种颜色之间人眼感知的差异,引入了颜色差异度量 $\Delta E$ (Delta E)。其中,$\Delta E_{2000}$ (CIE DE2000) 是目前最广泛接受的颜色差异公式,它在 Lab 空间的基础上进行了修正,以更好地反映人眼的非线性颜色感知特性,尤其是在中性色、亮度和色调方面。

\subsection[\hspace{-2pt}CIEDE2000色差公式]{{\heiti\zihao{-4} \hspace{-8pt}CIEDE2000色差公式}}\label{subsection2: CIEDE2000色差公式}

为了更精确的对问题进行建模并且便于后续损失函数以及差分进化算法的实现,我们将题目中的BT.2020颜色空间以及显示屏的颜色空间从 xy 色度坐标转换为 XYZ 颜色空间,再利用 Lab 颜色空间公式转换为 $(L^{*},a^{*},b^{*})$ 。最后计算 $\Delta E_{00}$ 损失值。

在将 BT.2020 高清视频源的色彩空间映射至普通显示屏 RGB 色域时,由于显示设备色域较小,无法完整覆盖原始色域,导致部分颜色无法被准确再现。因此,我们需要设计一个合理的\textbf{色彩转换映射矩阵} $M\in \mathbb{R}^{3\times 3}$ ,以最小化从 BT.2020 色域到显示屏色域的映射过程中所产生的\textbf{主观感知误差}。

为度量这一色彩差异,应选择符合人眼视觉感知的度量方式。传统的欧几里得差异(如 RGB 或 XYZ 空间中的 L2 距离)不能很好地反映颜色感知误差。我们引入国际照明委员会(CIE)推荐的 $\Delta E_{00}$ 作为感知误差的度量函数。

对任意两个颜色在Lab空间中的向量:
\begin{equation}
\begin{aligned}
  &Lab_{1} = (L^{*}_{1},a^{*}_{1},b^{*}_{1}),\ \ \ Lab_{2}=(L^{*}_{2},a^{*}_{2},b^{*}_{2})
\end{aligned}
\end{equation}

$\Delta E_{00}$ 的计算公式如下:
\begin{equation}
\begin{aligned}
  &\Delta E_{00}=\sqrt{(\frac{\Delta L^{'}}{k_{L}S_{L}})^{2}+(\frac{\Delta C^{'}}{k_{C}S_{C}})^{2}+(\frac{\Delta H^{'}}{k_{H}S_{H}})^{2}+R_{T}\cdot{(\frac{\Delta C^{'}}{k_{C}S_{C}})}\cdot (\frac{\Delta H^{'}}{k_{H}S_{H}})}
\end{aligned}
\end{equation}

其详细计算步骤包括:

\textbf{(1)明度差与平均明度}
\begin{equation}
\begin{aligned}
   &\Delta L^{'} = L^{*}_{2}-L^{*}_{1},\ \ \overline{L}=\frac{L^{*}_{2}-L^{*}_{1}}{2}
\end{aligned}
\end{equation}

\textbf{(2)色度差与平均色度}
\begin{equation}
\begin{aligned}
  &C_{1}=\sqrt{a^{*2}_{1}+b^{*2}_{1}},\ \ C_{2}=\sqrt{a^{*2}_{2}+b^{*2}_{2}},\ \ \Delta C^{'}=C_{2}-C_{1},\ \ \overline{C}=\frac{C_{1}+C_2}{2}
\end{aligned}
\end{equation}

\textbf{(3)色相角差与平均色相角}
\begin{equation}
\begin{aligned}
 &h_{1}=\arctan2(b^{*}_{1},a^{*}_{1}),\ \ h_{2}=\arctan2(b^{*}_{2},a^{*}_{2})\\
 &\Delta h^{'}=h_{2}-h_{1},\ \ \Delta H^{1}=2\sqrt{C_{1}C_{2}}\sin (\frac{\Delta h^{'}}{2})\\
 &\overline{h}=
 \begin{cases}
   \frac{h_{1}+h_{2}}{2},\ \ |h_{1}-h_{2}|>180^{\circ}\\
   \frac{h_{1}+h_{2}+360^{\circ}}{2},\ \ |h_{1}-h_{2}|\leq 180^{\circ}
 \end{cases}
\end{aligned}
\end{equation}

\textbf{(4)调整因子}
\begin{equation}
\begin{aligned}
  &G=0.5(1-\sqrt{\frac{\overline{C^{7}}}{\overline{C^{7}}+25^{7}}})\\
  &T=1-0.17\cos(\overline{h}-30^{\circ})+0.24\cos(2\overline{h})+0.32\cos(3\overline{h}+6^{\circ})-0.20\cos(4\overline{h}-63^{\circ})
\end{aligned}
\end{equation}

\textbf{(5)权重因子}
\begin{equation}
\begin{aligned}
 &S_{L}=1+\frac{0.015(L-50)^{2}}{\sqrt{20+(\overline{L}-50)^{2}}},\ \ S_{C}=1+0.045\overline{C},\ \ S_{H}=1+0.015\overline{C}T
\end{aligned}
\end{equation}

\textbf{(6)旋转补偿因子}
\begin{equation}
\begin{aligned}
 &R_{T}=-\sin(2\Delta \theta)\cdot R_{C},\ \ \Delta \theta=30\exp\{-(\frac{\overline{h}-275^{\circ}}{25})^{2}\},\ \ R_{C}=2\sqrt{\frac{\overline{C^{7}}}{\overline{C^{7}}+25^{7}}}
\end{aligned}
\end{equation}

其中:$\Delta L^{'}$ :明度差,$\Delta C^{'}$ :色度差,$\Delta H^{'}$ :色相差,$S_{L},S_{C},S_{H}$ :感知缩放因子,$k_{L}=k_{C}=k_{H}=1$ :常用单位权重。

\subsection[\hspace{-2pt}色差公式的应用意义]{{\heiti\zihao{-4} \hspace{-8pt}色差公式的应用意义}}\label{subsection2: 色差公式的应用意义}

由上述公式,可以计算出两个 CIELab值 的色差。该函数对人眼感知差异具有良好拟合性能,因此被广泛用于图像质量、颜色匹配等领域。$\Delta E_{2000}$ 值越小,表示两种颜色感知差异越小。通常认为:
\begin{itemize}
    \item $\Delta E_{00} < 1.0$:人眼难以察觉的颜色差异
    \item $1.0 \leq \Delta E_{00} < 2.0$:训练有素的观察者可察觉的微小差异
    \item $2.0 \leq \Delta E_{00} < 4.0$:普通观察者可察觉的明显差异
    \item $\Delta E_{00} \geq 4.0$:显著的颜色差异
\end{itemize}

\section[\hspace{-2pt}伽马校正理论]{{\heiti\zihao{4} \hspace{-8pt}伽马校正理论}}\label{section2: 伽马校正理论}

伽马校正(Gamma Correction)是数字图像处理中的重要概念,用于补偿显示设备的非线性响应特性。在理想情况下,显示设备的输出亮度应与输入信号成线性关系,但实际显示设备(如CRT显示器、LCD显示器、LED显示屏等)通常具有非线性的响应曲线。

\subsection[\hspace{-2pt}伽马响应模型]{{\heiti\zihao{-4} \hspace{-8pt}伽马响应模型}}\label{subsection2: 伽马响应模型}

对于显示设备的每个颜色通道$c \in \{R,G,B\}$,其响应特性可以用伽马函数建模:
\begin{equation}
I_{\text{output},c} = S_c \cdot (I_{\text{input},c})^{\gamma_c}
\end{equation}
其中:
\begin{itemize}
    \item $I_{\text{input},c}$:输入信号强度(归一化到[0,1]范围)
    \item $I_{\text{output},c}$:实际输出亮度(归一化到[0,1]范围)
    \item $\gamma_c$:伽马值,描述非线性程度
    \item $S_c$:比例因子,用于调整整体亮度水平
\end{itemize}

\subsection[\hspace{-2pt}伽马参数估计]{{\heiti\zihao{-4} \hspace{-8pt}伽马参数估计}}\label{subsection2: 伽马参数估计}

给定测量数据$I_{\text{meas},c}$和目标数据$I_{\text{target},c}$,可以通过对数线性回归估计伽马参数:
\begin{equation}
\log(I_{\text{meas},c}) = \gamma_c \log(I_{\text{target},c}) + \log(S_c)
\end{equation}

这可以转化为标准的线性回归问题:
\begin{equation}
\mathbf{y} = \mathbf{A}\boldsymbol{\theta} + \boldsymbol{\epsilon}
\end{equation}
其中$\mathbf{y} = \log(I_{\text{meas},c})$,$\mathbf{A} = [\log(I_{\text{target},c}), \mathbf{1}]$,$\boldsymbol{\theta} = [\gamma_c, \log(S_c)]^T$。

\subsection[\hspace{-2pt}伽马校正变换]{{\heiti\zihao{-4} \hspace{-8pt}伽马校正变换}}\label{subsection2: 伽马校正变换}

基于估计的伽马参数,可以定义前向和反向伽马校正变换:

\textbf{前向变换(编码):}
\begin{equation}
I_{\text{encoded}} = \mathrm{clip}((I_{\text{linear}} \cdot S)^{\gamma}, [0,1])
\end{equation}

\textbf{反向变换(线性化):}
\begin{equation}
I_{\text{linear}} = \mathrm{clip}((I_{\text{encoded}})^{1/\gamma} / S, [0,1])
\end{equation}

其中$\mathrm{clip}(\cdot, [0,1])$函数确保输出值在有效范围内。

\subsection[\hspace{-2pt}伽马校正的物理意义与应用]{{\heiti\zihao{-4} \hspace{-8pt}伽马校正的物理意义与应用}}\label{subsection2: 伽马校正的物理意义与应用}

伽马校正的物理意义体现在以下几个方面:

\begin{itemize}
    \item \textbf{设备特性补偿}:不同显示设备具有不同的伽马值,通过校正可以实现设备间的颜色一致性
    \item \textbf{感知均匀性}:人眼对亮度的感知是非线性的,适当的伽马校正可以更好地匹配人眼感知
    \item \textbf{动态范围优化}:伽马校正可以优化有限位深下的颜色表示,减少量化误差
\end{itemize}

在LED显示器颜色校正中,伽马校正是实现精确颜色还原的关键步骤,需要与线性矩阵变换相结合,形成完整的颜色校正流程。