\chapter[\hspace{0pt}问题重述]{{\heiti\zihao{3}\hspace{0pt}问题重述}}\label{chapter1: 问题重述}
d\cite{MAML}
\section[\hspace{-2pt}问题背景与意义]{{\heiti\zihao{-3} \hspace{-8pt}问题背景与意义}}\label{section1: 问题背景与意义}
随着当下显示器技术的发展,超高清技术、HDR技术的出现,显示器设备对色彩表现力的要求越来越高。然而,由于图像采集设备与图像现实设备二者对色彩的感知和还原能力存在差异,导致视频源中的色彩信息往往无法在显示设备上完美复现。在当前超高清显示器的需求日益增长的背景下,如何在有限色域的显示器中还原视频源的色彩,已成为高性能显示设备设计的关键难题。

国际照明委员会建立了标准色度学系统,这位颜色表达和转换提供了统一的数学框架。比如CIE1931色度图,可以将不同设备的色欲覆盖可视化。BT2020色彩空间是一种标准的高清视频源的三基色色空间。BT2020具有更广的色域范围,通常用于高动态范围视频何超高清电视的显示。而现实中普通显示屏色彩空间,诸如sRGB、NTSC等通常色域更小。这通常会导致部分高色彩饱和度区域无法准确重建,从而形成色彩损失问题。

针对上述问题,工业界提出多通道拓展方案,将视频源引入第四个颜色通道V(RGBV)拓宽了记录色域,而显示设备则拓展为五通道(RGBCX)以提升色彩重现能力。因而如何设计从高色域空间到显示器色域的映射函数,使得色彩失真最小,是提升色彩显示能力的关键任务。

此外被广泛使用的LED显示器,因其本身的制造差异、驱动电路的非线性影响等因素会导致整屏颜色显示不一致。这导致了显示效果的非一致性会严重影响视觉体验。因此基于颜色空间转换与匹配原理,合理构建映射函数和校正策略,对LED像素点颜色进行精细调控,从而实现整屏一致性的色彩校正,已成为提升LED显示品质的重要手段。

\section[\hspace{-2pt}问题提出与研究内容]{{\heiti\zihao{-3} \hspace{-8pt}问题提出与研究内容}}\label{section1: 问题提出与研究内容}

\subsection[\hspace{-2pt}问题一]{{\heiti\zihao{4} \hspace{-8pt}问题一}}\label{section1: 问题一}
本问题的核心在于实现不同色域之间的映射。BT2020色域更广,而sRGB色域相对较小。二者在色度坐标、亮度范围等方面存在较大差异,直接映射会导致显示器难以还原视频源的颜色,进而损失色彩,还会导致失真、亮度饱和度损失等问题。因此需要定义合适的转换损失函数,减小色彩损失。因此在映射过程中应当选择合适的损失函数,保证转换后的色彩贴合人眼视觉特性,提高感知效果。选择损失函数后还应当采用梯度下降法或基于样本的非线性最小二乘法进行求解。
