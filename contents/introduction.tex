\chapter[\hspace{0pt}问题重述与分析]{{\heiti\zihao{3}\hspace{0pt}问题重述与分析}}\label{chapter: 问题重述与分析}

%本章内容共分为四节,\hyperref[section1: 问题背景]{第一节}介绍本文的研究背景及意义;\hyperref[section1: 问题重述]{第二节}总结少样本分类算法的国内外研究现状,并对其面临的挑战进行分析;\hyperref[section1: 问题分析]{第三节}介绍本文的研究内容与创新点。

\section[\hspace{-2pt}问题背景]{{\heiti\zihao{-3} \hspace{-8pt}问题背景}}\label{section1: 问题背景}
%TODO
STR,也即短串联重复序列是法医DNA分析中最常用到的遗传标记。单一STR检测可以进行有效的辨别和区分;但混合STR分型却难以区分。国内目前依靠法医进行人工分析,精度与效率存在问题。//引用通过数学建模与计算方法实现混合STR智能化分析具有重要现实意义。

\section[\hspace{-2pt}问题分析]{{\heiti\zihao{-3} \hspace{-8pt}问题分析}}\label{section1: 问题分析}

\subsection[\hspace{-2pt}问题一]{{\heiti\zihao{4} \hspace{-8pt}问题一}}\label{section1: 问题一}
 确定混合样本中DNA贡献者的人数(Number of Contributors, NOC)是混合谱分析的第一步,也是后续步骤的前提。当下研究得出检测样本中的DNA模板量越大,则检测样本生成的STR图谱上的等位基因的峰面积越大///(基于全局最小残差法快速分析混合STR图谱)。因此我们可以通过峰面积推断可能的比例。但由于可能出现的等位基因重叠和漏检导致实际的峰数量和面积少于理论值。
 
 因此,我们提出基于峰高分布的高斯混合模型并结合信息准则的方法估计贡献者人数。我们假设每个贡献者的DNA含量不同并且峰高总值近似为高斯分布则N个人的集合可以建模为N个高斯分布的混合。我们对候选的不同人数假设  $ N=1,2,3,\dots $  进行GMM拟合,用EM算法估计混合模型参数(各高斯分布的均值、方差、权重等)。然后计算每个$N$假设下模型的赤池信息准则(AIC)或贝叶斯信息准则(BIC)。信息准则综合考虑模型拟合优度和复杂度,值越低表示平衡后效果越好。选择使AIC/BIC最小的$N$作为估计的贡献者人数。

\subsection[\hspace{-2pt}问题二]{{\heiti\zihao{4} \hspace{-8pt}问题二}}\label{section1: 问题二}
确定了混合样本中的贡献者人数后,我们需要估计每位贡献者的DNA相对含量,即混合比例向量 $\mathbf{r}=(r\_1,\dots,r\_N)$。然而当比例向量相对接近时,等位基因峰高将高度重叠,难以正确区分等位基因。因此我们需要建立模型将等位基因峰高表达为多个贡献体的叠加并求解比例。

我们采用////
\subsection[\hspace{-2pt}问题三]{{\heiti\zihao{4} \hspace{-8pt}问题三}}\label{section1: 问题三}
第三个子问题要求我们将混合的STR图谱拆解还原出每个贡献者各自的基因型,并评估准确性。这实际上是对混合DNA进行个体分型的过程。如果有一个已知的嫌疑人/个体基因型数据库(附件3提供),则还需要将分离得到的基因型与数据库中的个体进行匹配,从而识别出混合样本中都有哪些人。此问题是整个混合谱分析的核心与难点:我们需要在不知道哪条峰属于哪个人的情况下,将彼此交织的等位基因峰正确归属到各个人的两个等位基因序列中去。对于一个$N$人混合、每人基因座数为$L$的情况,可能的基因型组合空间非常巨大(每个位点需要在观测峰中为$N$个人各选择2个等位基因,组合数爆炸式增长)。因此必须设计高效算法进行合理搜索和判断。

我们提出////
\subsection[\hspace{-2pt}问题四]{{\heiti\zihao{4} \hspace{-8pt}问题四}}\label{section1: 问题四}
混合STR图谱中经常伴随各种实验噪声和假信号,如果不加以处理,可能干扰前述分析步骤的准确性。常见的噪声包括:电泳基线噪声导致的微小峰等。此外,还有信号本身的随机波动。若直接将所有观测峰都视为真实等位基因,将导致贡献者人数高估或错误的基因型分配。因此,需要在混合图谱分析中引入降噪机制,提高信噪比。

对此我们采用////
