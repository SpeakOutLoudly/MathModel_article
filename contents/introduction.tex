\chapter[\hspace{0pt}问题重述]{{\heiti\zihao{3}\hspace{0pt}问题重述}}\label{chapter1: 问题重述}

\removelofgap
\removelotgap

\section[\hspace{-2pt}问题背景与意义]{{\heiti\zihao{-3} \hspace{-8pt}问题背景与意义}}\label{section1: 问题背景与意义}

走在晚风都市,或春日田野,我们都会看到一个色彩斑斓的世界。色彩是我们对世界一种重要感知。颜色是光作用于人眼引起的视觉感知现象,它与物体的材质和光照有关,由物体反射(或发射)的电磁波特定波长决定,其本质是大脑对光刺激的神经信号处理的结果。

为了更好地复原现实世界的色彩,我们需要色彩采集设备(光谱色差仪、摄像机等)和显示设备(显示器等)。由于颜色记录设备(如摄像机)的感知能力和颜色显示设备(如LED显示器)的还原能力不完全一致,如何将记录设备的颜色逼真表达出来是高性能显示器的主要目标。

根据人眼视觉的特性,在彩色复现过程中,重要的是获得与原景物相同的彩色感觉,并不要求完全恢复原景物辐射光的光谱成分;而与某一颜色相同的彩色感觉,可由不同光谱分布的色光组合产生。如果适当选择三基色,将它们按不同比例进行合成,就可以引起各种不同的色彩感觉,合成彩色的亮度由这三个基色的亮度相加之和决定,而色度则由三个基色分量的比例决定。

根据三基色原理,可以采用不同的三色组合。但是在显示领域中,比较多的采用红色(R)、绿色(G)和蓝色(B)的光谱区域内选择三个基色。这样自然界中所能观察到的各种颜色,几乎都能由它们合成出来。

\subsection[\hspace{-2pt}CIE标准色度学系统]{{\heiti\zihao{4} \hspace{-8pt}CIE标准色度学系统}}\label{subsection1: CIE标准色度学系统}

现代色度学采用国际照明委员会(CIE)所规定的一套颜色测量原理、数据和计算方法,称为CIE标准色度学系统。

\noindent\textbf{(1)1931CIE-RGB颜色系统}

1931年CIE在2°视场条件下,进行了专门的颜色混合匹配实验,定出匹配等能光谱色的$\overline{r}(\lambda)$、$\overline{g}(\lambda)$、$\overline{b}(\lambda)$($\lambda$为光谱波长)光谱三刺激函数,这三个函数即上述中提到的大脑对光刺激神经信号的数学表达式,称为"1931 CIE-RGB系统标准色度观察者光谱三刺激值",简称为"1931 CIE-RGB光谱三刺激值"。系统采用波长为700nm的红、546.1nm的绿和435.8nm的蓝作为(R)、(G)、(B)三原色。它们为色度学奠定了数学基础。

\noindent\textbf{(2)1931CIE-XYZ颜色系统}

1931CIE-XYZ颜色系统是在1931CIE-RGB颜色系统的基础上,用数学方法,选用三个理想的三原色(X)、(Y)、(Z),将1931CIE-RGB系统中的光谱三刺激值$\overline{r}(\lambda)$、$\overline{g}(\lambda)$、$\overline{b}(\lambda)$和色度坐标r、g、b均变为正值。三原色(X)代表红原色,(Y)代表绿原色,(Z)代表蓝原色。在(X)、(Y)、(Z)系统上三原色$x$、$y$、$z$与1931CIE-RGB系统的坐标$r$、$g$、$b$可以互相转换。通过线性变换成XYZ空间,再做归一化之后,最终便出现了马蹄形的CIE1931标准色空间。

随着当下显示器技术的发展,超高清技术、HDR技术的出现,显示器设备对色彩表现力的要求越来越高。然而,由于图像采集设备与图像显示设备二者对色彩的感知和还原能力存在差异,导致视频源中的色彩信息往往无法在显示设备上完美复现。在当前超高清显示器的需求日益增长的背景下,如何在有限色域的显示器中还原视频源的色彩,已成为高性能显示设备设计的关键难题。

BT2020色彩空间是一种标准的高清视频源的三基色色空间,具有更广的色域范围,通常用于高动态范围视频和超高清电视的显示。而现实中普通显示屏色彩空间,诸如sRGB、NTSC等通常色域更小。这通常会导致部分高色彩饱和度区域无法准确重建,从而形成色彩损失问题。针对上述问题,工业界提出多通道拓展方案,将视频源引入第四个颜色通道V(RGBV)拓宽了记录色域,而显示设备则拓展为五通道(RGBCX)以提升色彩重现能力。

此外,被广泛使用的LED显示器,因其本身的制造差异、驱动电路的非线性影响等因素会导致整屏颜色显示不一致。这导致了显示效果的非一致性会严重影响视觉体验。因此基于颜色空间转换与匹配原理,合理构建映射函数和校正策略,对LED像素点颜色进行精细调控,从而实现整屏一致性的色彩校正,已成为提升LED显示品质的重要手段。

\section[\hspace{-2pt}问题提出与研究内容]{{\heiti\zihao{-3} \hspace{-8pt}问题提出与研究内容}}\label{section1: 问题提出与研究内容}

如前面提到,在现实中颜色的显示设备的表达能力与记录设备的颜色感知能力并不完全一致,如何在现有显示能力下更好的表达记录的图像(或视频)是显示器颜色工程(如LED显示器颜色设计)的重要任务。

\subsection[\hspace{-2pt}问题一:颜色空间转换]{{\heiti\zihao{4} \hspace{-8pt}问题一:颜色空间转换}}\label{subsection1: 问题一}

CIE1931的标准色空间为马蹄形状,自然界中我们所观察到的所有颜色坐标都可以表示在这个马蹄形状的曲线内,每个坐标值表示的便是一种颜色。

在CIE1931色度图中,棕色三角形表示BT2020标准的高清视频源的三基色色空间,而红色三角形表示的通常普通显示屏的RGB三基色空间。红色三角形所形成的色域比棕色的小,所表示出的颜色就比较少,显示器不能完全还原出视频源记录的颜色,从而导致色彩损失,但这是不可避免的。

本问题的核心在于实现不同色域之间的映射。BT2020色域更广,而sRGB色域相对较小。二者在色度坐标、亮度范围等方面存在较大差异,直接映射会导致显示器难以还原视频源的颜色,进而损失色彩,还会导致失真、亮度饱和度损失等问题。

\textbf{要求}:试定义合适的转换损失函数,设计视频源颜色空间到显示屏RGB颜色空间的转换映射,使色彩转换损失最小。在映射过程中应当选择合适的损失函数,保证转换后的色彩贴合人眼视觉特性,提高感知效果。

\subsection[\hspace{-2pt}问题二:颜色空间转换(4通道到5通道)]{{\heiti\zihao{4} \hspace{-8pt}问题二:颜色空间转换(4通道到5通道)}}\label{subsection1: 问题二}

为了最大程度的呈现大自然界的中颜色,通常将摄像机增加了一个颜色通道$V:(Y_V,x_V,y_V)$,即摄像机可以输出四基色视频源RGBV,从而扩大了色域空间的面积。四通道的坐标(包含亮度信号)分别为:
\begin{equation}
\begin{cases}
R:(Y_R,x_R,y_R) \\
G:(Y_G,x_G,y_G) \\
B:(Y_B,x_B,y_B) \\
V:(Y_V,x_V,y_V)
\end{cases}
\end{equation}

这里,$Y_S$($S$表示R、G、B或V)为亮度信息。类似地,为了增强LED的显示能力,也可以设计成为五基色(通道)的显示屏RGBCX,形成五边形的色域范围。

\textbf{要求}:试定义合适的颜色转换映射,将视频源4通道信号转化到五通道LED显示器上,使颜色转换损失最小。

\subsection[\hspace{-2pt}问题三:LED显示器颜色校正]{{\heiti\zihao{4} \hspace{-8pt}问题三:LED显示器颜色校正}}\label{subsection1: 问题三}

由于组成彩色LED全显示屏(如分辨率1920×1080)每个像素的发光器件内部色度存在差异,全彩LED模块显示屏的颜色即使全都在同样的标定值(220)下,呈现的色彩也会有差异。从校正前的LED显示屏成像结果可以看出显示不一致,不能满足高品质的显示需求。

因此,我们需要利用颜色的合成特性将颜色进行校正,使显示器在标定值(220)下呈现均匀一致的R、G、B颜色输出效果。

\textbf{要求}:试根据设计的(1)-(2)色域转换结果应用在LED颜色校正中,将全屏颜色进行校正并运用在给定的64×64的显示数据模块上。

\textbf{数据说明}:附件提供64×64×10数据集合(注:包括显示的目标值(每个像素设定为220)和每个受扰动的屏幕显示的R、G、B值)。
