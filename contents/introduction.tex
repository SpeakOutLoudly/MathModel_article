\chapter[\hspace{0pt}问题重述与分析]{{\heiti\zihao{3}\hspace{0pt}问题重述与分析}}\label{chapter1: 问题重述与分析}

\removelofgap
\removelotgap

\section[\hspace{-2pt}问题背景与意义]{{\heiti\zihao{-3} \hspace{-8pt}问题背景与意义}}\label{section1: 问题背景与意义}

随着超高清技术、HDR技术的发展,显示器设备对色彩表现力要求越来越高。然而,由于图像采集设备与显示设备对色彩的感知和还原能力存在差异,导致视频源色彩信息往往无法完美复现。BT2020色彩空间具有更广的色域范围,而sRGB、NTSC等普通显示屏色域较小,导致部分高饱和度区域无法准确重建。为此,工业界提出多通道拓展方案:将视频源引入第四个颜色通道V(RGBV)拓宽记录色域,显示设备拓展为五通道(RGBCX)提升色彩重现能力。此外,LED显示器因制造差异、驱动电路非线性等因素导致整屏颜色显示不一致,严重影响视觉体验。因此,基于颜色空间转换与匹配原理,构建映射函数和校正策略,对LED像素点进行精细调控,实现整屏一致性色彩校正,已成为提升LED显示品质的重要手段。

\section[\hspace{-2pt}问题提出与研究内容]{{\heiti\zihao{-3} \hspace{-8pt}问题提出与研究内容}}\label{section1: 问题提出与研究内容}

如前面提到,在现实中颜色的显示设备的表达能力与记录设备的颜色感知能力并不完全一致,如何在现有显示能力下更好的表达记录的图像(或视频)是显示器颜色工程(如LED显示器颜色设计)的重要任务。

\subsection[\hspace{-2pt}问题一:颜色空间转换]{{\heiti\zihao{4} \hspace{-8pt}问题一:颜色空间转换}}\label{subsection1: 问题一}

CIE1931的标准色空间为马蹄形状,自然界中我们所观察到的所有颜色坐标都可以表示在这个马蹄形状的曲线内,每个坐标值表示的便是一种颜色。

在CIE1931色度图中,棕色三角形表示BT2020标准的高清视频源的三基色色空间,而红色三角形表示的通常普通显示屏的RGB三基色空间。红色三角形所形成的色域比棕色的小,所表示出的颜色就比较少,显示器不能完全还原出视频源记录的颜色,从而导致色彩损失,但这是不可避免的。

本问题的核心在于实现不同色域之间的映射。BT2020色域更广,而sRGB色域相对较小。二者在色度坐标、亮度范围等方面存在较大差异,直接映射会导致显示器难以还原视频源的颜色,进而损失色彩,还会导致失真、亮度饱和度损失等问题。

\textbf{要求}:试定义合适的转换损失函数,设计视频源颜色空间到显示屏RGB颜色空间的转换映射,使色彩转换损失最小。在映射过程中应当选择合适的损失函数,保证转换后的色彩贴合人眼视觉特性,提高感知效果。

\subsection[\hspace{-2pt}问题二:颜色空间转换(4通道到5通道)]{{\heiti\zihao{4} \hspace{-8pt}问题二:颜色空间转换(4通道到5通道)}}\label{subsection1: 问题二}

为了最大程度的呈现大自然界的中颜色,通常将摄像机增加了一个颜色通道$V:(Y_V,x_V,y_V)$,即摄像机可以输出四基色视频源RGBV,从而扩大了色域空间的面积。四通道的坐标(包含亮度信号)分别为:
\begin{equation}
\begin{cases}
R:(Y_R,x_R,y_R) \\
G:(Y_G,x_G,y_G) \\
B:(Y_B,x_B,y_B) \\
V:(Y_V,x_V,y_V)
\end{cases}
\end{equation}

这里,$Y_S$($S$表示R、G、B或V)为亮度信息。类似地,为了增强LED的显示能力,也可以设计成为五基色(通道)的显示屏RGBCX,形成五边形的色域范围。

\textbf{要求}:试定义合适的颜色转换映射,将视频源4通道信号转化到五通道LED显示器上,使颜色转换损失最小。

\subsection[\hspace{-2pt}问题三:LED显示器颜色校正]{{\heiti\zihao{4} \hspace{-8pt}问题三:LED显示器颜色校正}}\label{subsection1: 问题三}

由于组成彩色LED全显示屏(如分辨率1920×1080)每个像素的发光器件内部色度存在差异,全彩LED模块显示屏的颜色即使全都在同样的标定值(220)下,呈现的色彩也会有差异。从校正前的LED显示屏成像结果可以看出显示不一致,不能满足高品质的显示需求。

因此,我们需要利用颜色的合成特性将颜色进行校正,使显示器在标定值(220)下呈现均匀一致的R、G、B颜色输出效果。

\textbf{要求}:试根据设计的(1)-(2)色域转换结果应用在LED颜色校正中,将全屏颜色进行校正并运用在给定的64×64的显示数据模块上。

\textbf{数据说明}:附件提供64×64×10数据集合(注:包括显示的目标值(每个像素设定为220)和每个受扰动的屏幕显示的R、G、B值)。
