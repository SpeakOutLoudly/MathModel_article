\chapter[致\hskip\ccwd{}\hskip\ccwd{}谢]{{\heiti\zihao{3}致\hskip\ccwd{}\hskip\ccwd{}谢}}

% 这里用盲审环境包裹致谢,在开启盲审开关时,环境内部的内容不予渲染。
% \begin{secretizeEnv}

转眼间,从本科到硕士,已在重庆度过了七年,我的求学之路也告一段落。临近道别之际,借此机会给每一个支持和帮助过我的人们说一声感谢。

首先,感谢我的导师黄晟教授。在整个研究生阶段,导师给予了我无微不至的指导和关怀。是他的悉心教导,让我不断进步,最终完成了这篇毕业论文。在我遇到困难和挑战时,导师总是耐心倾听、指导并鼓励我,帮助我面对困难。在此,我向导师黄晟教授表示最诚挚的感谢!

同时,感谢重庆大学为我们提供了良好的学习环境和条件。学校的丰富资源和先进设施为我的研究提供了有力支持,让我能够顺利进行实验和调研。在这里,我结识了许多优秀的同学和朋友,他们的讨论和交流激发了我的灵感,也让我收获了很多。

另外,感谢我的家人。无论是在生活中还是在学业上,他们始终是我坚强的后盾,一直默默支持着我,使我能够安心学习。感谢我的女朋友魏欣钰女士,与我分享快乐,遇到挫折时给予我鼓励,帮助我渡过难关,不断前行。

感谢所有在论文写作过程中帮助过我的老师、同学和朋友们。尤其感谢张译、周锋涛、杨万里师兄入学以来的指导以及陈忠明、唐文浩、何涛同门提供的诸多帮助,是你们的建议、讨论和帮助让我不断改进论文,使其更加完善。在我即将踏上新征程之际,我会铭记你们的帮助和支持。

最后,感谢百忙之中参与评阅和答辩的各位专家、教授。

% \vfill
\vspace*{2em}
\begin{flushright}
{\CJKfontspec{STXingkai} \Large 尹国伟} \hspace*{3.5em}
\\  \hspace*{\fill} \\
{二〇二四年五月\hspace*{1em}于重庆}
\end{flushright}
% \end{secretizeEnv}