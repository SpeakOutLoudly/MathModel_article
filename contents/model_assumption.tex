\chapter[\hspace{0pt}模型假设]{{\heiti\zihao{3}\hspace{0pt}模型假设}}\label{chapter2:模型假设}

\removelofgap
\removelotgap

本文针对LED显示器颜色转换与校正问题建立的数学模型基于以下核心假设:

\section[\hspace{-2pt}颜色空间线性映射假设]{{\heiti\zihao{-3} \hspace{-8pt}颜色空间线性映射假设}}\label{section2: 颜色空间线性映射假设}

\textbf{假设1}:BT.2020色域到显示屏RGB色域的转换可以通过$3\times 3$线性变换矩阵$M$精确描述,即:
$$c' = M \cdot c$$
其中$c$为BT.2020空间下的颜色向量,$c'$为目标显示空间下的颜色向量。

该假设忽略了色域边界附近可能存在的非线性效应和色彩适应现象,适用于大部分常规显示内容的颜色转换。

\section[\hspace{-2pt}设备响应稳定性假设]{{\heiti\zihao{-3} \hspace{-8pt}设备响应稳定性假设}}\label{section2: 设备响应稳定性假设}

\textbf{假设2}:相机和LED显示屏的颜色响应特性在建模时间窗口内保持稳定,不考虑设备老化、温度漂移等因素的影响。

\textbf{假设3}:LED显示器各像素点具有空间均匀性,即同一输入信号在不同位置产生相同的颜色输出,忽略制造工艺导致的像素间差异。

这些假设简化了时变和空间变化的复杂性,使模型能够专注于核心的颜色转换机理。

\section[\hspace{-2pt}观察环境理想化假设]{{\heiti\zihao{-3} \hspace{-8pt}观察环境理想化假设}}\label{section2: 观察环境理想化假设}

\textbf{假设4}:颜色评估在标准观察条件下进行,包括:
\begin{itemize}
    \item 标准照明体D65作为参考白点
    \item 观察角度为正视角(0°)
    \item 环境光照稳定,无杂散光干扰
\end{itemize}

该假设排除了复杂环境因素对颜色感知的影响,使色差计算基于标准化条件。

\section[\hspace{-2pt}伽马响应模型假设]{{\heiti\zihao{-3} \hspace{-8pt}伽马响应模型假设}}\label{section2: 伽马响应模型假设}

\textbf{假设5}:LED显示器的电光转换特性遵循简化的伽马响应模型:
$$I_{\text{output}} = S \cdot (I_{\text{input}})^{\gamma}$$
其中$\gamma$为伽马值,$S$为比例因子。

该假设忽略了复杂的电路非线性和光学串扰效应,适用于建立一阶近似的校正模型。

\section[\hspace{-2pt}数据代表性假设]{{\heiti\zihao{-3} \hspace{-8pt}数据代表性假设}}\label{section2: 数据代表性假设}

\textbf{假设6}:训练样本和测试数据能够充分代表实际应用中的颜色分布,模型的泛化性能不受数据偏差影响。

该假设对神经网络模型尤为重要,确保了从有限样本学习到的映射关系能够推广到未见过的颜色组合。

以上假设构成了本文模型体系的理论基础,在实际应用中应根据具体条件评估假设的适用性。

