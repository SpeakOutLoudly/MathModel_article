\cqusetup{
%	************	注意	************
%	* 1. \cqusetup{}中不能出现全空的行,如果需要全空行请在行首注释
%	* 2. 不需要的配置信息可以放心地坐视不理、留空、删除或注释(都不会有影响)
%	*
%	********************************
% ===================
%	论文的中英文题目
% ===================
  ctitle = {待补充},
  etitle = {待补充},
% ===================
% 作者部分的信息
% \secretize{}为盲审标记点,在打开盲审开关时内容会自动被替换为***输出,盲审开关默认关闭
% ===================
  cauthor = \secretize{尹国伟},	% 你的姓名,以下每项都以英文逗号结束
  eauthor = \secretize{Guowei~Yin},	% 姓名拼音,~代表不会断行的空格
  studentid = \secretize{},	% 仅本科生,学号
  csupervisor = \secretize{黄~~~晟~~~~~教授},	% 导师的姓名
  esupervisor = \secretize{{Prof.~Sheng Huang}},	% 导师的姓名拼音
  cassistsupervisor = \secretize{}, % 本科生可选,助理指导教师姓名,不用时请留空为{}
  cextrasupervisor = \secretize{}, % 本科生可选,校外指导教师姓名,不用时请留空为{}
  eassistsupervisor = \secretize{}, % 本科生可选,助理指导教师或/和校外指导教师姓名拼音,不用时请留空为{}
  cpsupervisor = \secretize{}, % 仅专硕,兼职导师姓名
  epsupervisor = \secretize{},	% 仅专硕,兼职导师姓名拼音
  cclass = \secretize{\rmfamily{2025}\heiti{年}\rmfamily{5}\heiti{月}},	% 博士生和学硕填学科门类,学硕填学科类型
  research_direction = \zihao{3}{工学},
  edgree = {},	% 专硕填Professional Degree,其他按实情填写
% % 提示:如果内容太长,可以用\zihao{}命令控制字号,作用范围:{}内
  cmajor = 工~~~~学,	% 专硕不需填,填写专业名称
  emajor = , % % 专硕不需填,填写专业英文名称
  cmajora = \zihao{3}{软件工程},
  cmajorb = \zihao{3}{计算机视觉},
  cmajorc = \secretize{},
  % cmajord = 2024年6月,
% ===================
% 底部的学院名称和日期
% ===================
  cdepartment = ,	%学院名称
  edepartment = ,	%学院英文名称
% ===================
% 封面的日期可以自动生成(注释掉时),也可以解除注释手动指定,例如:二〇一六年五月
% ===================
%	mycdate = {2023年6月},
%	myedate = {June 2023},
}% End of \cqusetup
% ===================
%
% 论文的摘要
%
% ===================
\begin{cabstract}	% 中文摘要
色彩是我们感知斑斓世界的重要方式,如何在数字设备中准确采集并逼真再现这些色彩,是色彩科学与显示技术领域的核心挑战与持续追求。为此,本文针对题目中的问题采用颜色空间、色域转换等相关理论,利用差分进化算法以及神经网络等优化算法,建立数学模型进行分析并求解,最终得到最佳的色域转换、多通道颜色系统转换以及LED显示器校正模型。

针对问题一,首先将题目中 BT.2020 和显示屏色域(sRGB)的色度坐标转换为"设备无关"的 XYZ 三刺激值,并基于此三刺激值转换为 CIELab 空间。我们选择专业的广泛使用的 $\Delta E_{00}$ 色差公式,并以其为优化目标,使用差分进化算法对其进行优化。最后得到视频源色域到显示屏色域的 XYZ 三刺激值转换矩阵。将转换后的结果与原显示屏色域相比可以得到均值 0.0744 的 $\Delta E_{00}$ 损失值以及最大值小于 0.001 的色度图面积差值。说明此模型保证了极低的感知误差以及对目标色域的高保真拟合。

针对问题二,我们通过训练一个神经网络来学习从四通道RGBV输入到五通道RGBCX输出的映射关系。为此,设计了一个 ColorNet,其输入层接收4个特征(RGBV),输出层产生5个特征(RGBCX)。训练过程使用了一个由受控非线性变换生成的数据集。我们采用结合均方误差(MSE)以及 $\Delta E_{00}$ 的加权损失函数,保证整体通道一致性并考虑感知上色彩的准确复现。最终加权损失值在0.4-0.7范围内,进一步验证了模型的有效性,并且没有出现明显的过拟合现象。模型实现了更广阔的色域覆盖与精准映射以及颜色失真最小化。

针对问题三,建立了基于CIE Lab色彩空间和三基色原理的LED显示器颜色校正模型。该模型结合伽马校正与线性矩阵变换,采用差分进化算法优化校正参数,实现精确的颜色还原。首先,通过对数线性回归估计LED显示器的伽马参数,发现LED显示器在不同颜色通道上的非线性响应特性具有差异性。然后,设计包含色差损失、正则化项和行列式惩罚的综合目标函数,基于CIE $\Delta E_{00}$色差公式构建优化目标。采用全局-局部混合优化策略:先用差分进化算法进行全局搜索,再用L-BFGS-B方法进行局部精调。实验结果表明,三种基色图像的平均色差从2.0以上降低到0.1左右,平均改善幅度达95.6\%;校正后100\%像素的色差均小于1.0,达到人眼难以察觉的优秀标准;校正矩阵行列式值约0.10,保证了数值稳定性和变换可逆性。
\end{cabstract}
% 中文关键词,请使用英文逗号分隔:
\ckeywords{颜色空间转换;LED显示器校正;差分进化;神经网络;伽马校正}

% 封面和摘要配置完成