\cqusetup{
%	************	注意	************
%	* 1. \cqusetup{}中不能出现全空的行,如果需要全空行请在行首注释
%	* 2. 不需要的配置信息可以放心地坐视不理、留空、删除或注释(都不会有影响)
%	*
%	********************************
% ===================
%	论文的中英文题目
% ===================
  ctitle = {待补充},
  etitle = {待补充},
% ===================
% 作者部分的信息
% \secretize{}为盲审标记点,在打开盲审开关时内容会自动被替换为***输出,盲审开关默认关闭
% ===================
  cauthor = \secretize{尹国伟},	% 你的姓名,以下每项都以英文逗号结束
  eauthor = \secretize{Guowei~Yin},	% 姓名拼音,~代表不会断行的空格
  studentid = \secretize{},	% 仅本科生,学号
  csupervisor = \secretize{黄~~~晟~~~~~教授},	% 导师的姓名
  esupervisor = \secretize{{Prof.~Sheng Huang}},	% 导师的姓名拼音
  cassistsupervisor = \secretize{}, % 本科生可选,助理指导教师姓名,不用时请留空为{}
  cextrasupervisor = \secretize{}, % 本科生可选,校外指导教师姓名,不用时请留空为{}
  eassistsupervisor = \secretize{}, % 本科生可选,助理指导教师或/和校外指导教师姓名拼音,不用时请留空为{}
  cpsupervisor = \secretize{}, % 仅专硕,兼职导师姓名
  epsupervisor = \secretize{},	% 仅专硕,兼职导师姓名拼音
  cclass = \secretize{\rmfamily{2025}\heiti{年}\rmfamily{5}\heiti{月}},	% 博士生和学硕填学科门类,学硕填学科类型
  research_direction = \zihao{3}{工学},
  edgree = {},	% 专硕填Professional Degree,其他按实情填写
% % 提示:如果内容太长,可以用\zihao{}命令控制字号,作用范围:{}内
  cmajor = 工~~~~学,	% 专硕不需填,填写专业名称
  emajor = , % % 专硕不需填,填写专业英文名称
  cmajora = \zihao{3}{软件工程},
  cmajorb = \zihao{3}{计算机视觉},
  cmajorc = \secretize{},
  % cmajord = 2024年6月,
% ===================
% 底部的学院名称和日期
% ===================
  cdepartment = ,	%学院名称
  edepartment = ,	%学院英文名称
% ===================
% 封面的日期可以自动生成(注释掉时),也可以解除注释手动指定,例如:二〇一六年五月
% ===================
%	mycdate = {2023年6月},
%	myedate = {June 2023},
}% End of \cqusetup
% ===================
%
% 论文的摘要
%
% ===================
\begin{cabstract}	% 中文摘要
针对问题一,首先将题目中 BT.2020 和显示屏色域(sRGB)的色度坐标转换为“设备无关”的 XYZ 三刺激值,并基于此三刺激值转换为 CIELab 空间。我们选择专业的广泛使用的 $\Delta E_{2000}$ 色差公式,并以其为优化目标,使用差分进化算法对其进行优化。最后得到视频源色域到显示屏色域的 XYZ 三刺激值转换矩阵。将转换后的结果与原显示屏色域相比可以得到极低的 $\Delta E_{2000}$ 损失值以及色度图面积差值。说明此模型具有极低的感知误差以及对目标色域的高保真拟合。

针对问题二,我们通过训练一个神经网络来学习从四通道RGBV输入到五通道RGBCX输出的映射关系。为此,设计了一个 ColorNet(一个简单的多层感知器),其输入层接收4个特征(RGBV),输出层产生5个特征(RGBCX)。训练过程使用了一个合成数据集,该数据集通过受控的非线性变换生成,以模拟现实世界的复杂性。最小化颜色转换损失的关键在于自定义的 CombinedLoss 函数。该函数结合了两种损失成分:均方误差(MSE),$\Delta E_{2000}$,通过对这两种损失成分进行加权,模型在保持整体通道一致性的同时,优先考虑感知上准确的色彩再现。模型使用AdamW优化器进行训练,并通过跟踪训练损失和验证损失来评估其性能。最后,解决方案包括ΔE2000误差分布、输入和输出系统色域的色度图以及样本颜色预测的可视化,以评估学习到的映射的有效性。

针对问题三
\end{cabstract}
% 中文关键词,请使用英文逗号分隔:
\ckeywords{;;差分进化;神经网络}


% 封面和摘要配置完成