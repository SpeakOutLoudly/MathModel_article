\chapter[\hspace{0pt}基于语义-视觉多空间关系建模的少样本分类研究]{{\heiti\zihao{3}\hspace{0pt}基于语义-视觉多空间关系建模的少样本分类研究}}\label{chapter4: 基于语义-视觉多空间关系建模的少样本分类研究}
\removelofgap
\removelotgap

上一章研究了基于多粒度样本关系建模的少样本特征学习算法,通过充分挖掘不同粒度的样本关系提高了网络所提取特征的质量。然而,该算法仅在视觉空间对多种样本关系进行了建模,忽略了数据集中所隐含的丰富语义信息,限制了模型通过基类数据进行训练来学习新类数据知识的能力。因此,在上一章的基础上,本章主要研究基于语义-视觉多空间关系建模的少样本特征适配算法,通过引入语义信息并与视觉信息进行建模从而丰富模型所获得的信息,增强模型的泛化能力。本章内容共分为四节,\hyperref[section4: 引言]{第一节}介绍研究动机和方法概述;\hyperref[section4: 基于语义-视觉多空间关系建模的少样本特征适配算法]{第二节}介绍本章提出的基于语义-视觉多空间关系建模的少样本特征适配算法;\hyperref[section4: 实验设置及结果分析]{第三节}给出实验设置和结果分析;\hyperref[section4: 本章小结]{第四节}对本章进行小结。

\section[\hspace{-2pt}引言]{{\heiti\zihao{-3} \hspace{-8pt}引言}}\label{section4: 引言}

\section[\hspace{-2pt}基于语义-视觉多空间关系建模的少样本特征适配算法]{{\heiti\zihao{-3} \hspace{-8pt}基于语义-视觉多空间关系建模的少样本特征适配算法}}\label{section4: 基于语义-视觉多空间关系建模的少样本特征适配算法}

\section[\hspace{-2pt}实验设置及结果分析]{{\heiti\zihao{-3} \hspace{-8pt}实验设置及结果分析}}\label{section4: 实验设置及结果分析}

\section[\hspace{-2pt}本章小结]{{\heiti\zihao{-3} \hspace{-8pt}本章小结}}\label{section4: 本章小结}

本章研究基于语义-视觉多空间关系建模的少样本特征适配算法,针对少样本分类中仅根据少量视觉特征无法捕获类别代表性特征的缺点,引入语义信息作为视觉信息的补充,通过对语义-视觉多空间关系进行建模,提出了语义-视觉多空间映射适配模型(Semantic-Visual Multi-Space Mapping Adapter,简称SVMSMA),以丰富样本特征的信息来源,利用语义特征对视觉特征进行补充与修正,从而提升模型在新类上的泛化能力。SVMSMA模型使用单/多模态映射网络将样本语义特征映射到视觉空间获得单/多模态映射特征,并通过跨模态分类(CMC)模块与跨模态特征对齐(CMFA)模块对映射网络进行优化,以使得语义特征与视觉特征建立联系。测试过程中,本章方法将支持集的视觉特征、单模态映射特征、以及多模态映射特征共同作为分类器的训练数据,达到了较仅使用单一特征时更好的分类结果。在miniImageNet、tieredImageNet、CIFAR-FS和CUB-200-2011数据集的大量实验表明了SVMSMA方法的有效性。

综上所述,本章提出的基于语义-视觉多空间关系建模的少样本特征适配算法通过对语义-视觉多空间关系进行建模,充分利用语义信息对视觉信息进行了补充,丰富了样本特征信息来源,提升了模型的泛化能力。
